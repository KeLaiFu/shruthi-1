\documentclass[a4paper,11pt]{article}
\usepackage[utf8]{inputenc}
\usepackage[T1]{fontenc}
\usepackage[english]{babel}
\usepackage{times}
\usepackage{graphicx}
\textwidth=6in
\textheight=9.0in
\headheight=0in
\headsep=0in
\oddsidemargin=0in
\evensidemargin=0in
\title{SSM2164 SVF}
\author{Emilie Gillet -- \tt emilie.o.gillet@gmail.com}
\date{}
\begin{document}

\maketitle

\includegraphics[width=\textwidth]{svf_schematics.pdf}

\section{Theory}

Notations:

$R_i$ is the value of the resistor through which the input signal is fed to the circuit.

$R_g$ is the value of the resistors through which the HP and LP outputs are fed back into the input.

$R_q$ is the value of the resistor through which the attenuated BP output is fed back into the input.

$R$ is the value of the resistors through which input voltages are converted into currents at the input of the 2164s, and through which the current at the output of the Q attenuator is converted back into a voltage.

$C$ is the value of the integrators' capacitors.

$v_{cv}$ is the cutoff frequency control voltage and $v_{q}$ is the reso control voltage.


The input voltage is $v_i(s)(s)$. $v_{lp}(s)$, $v_{hp}(s)$, $v_{bp}(s)$ are respectively the voltages at the low-pass, high-pass and band-pass nodes of this circuit.

As a reminder, the transfer function of a SSM2164 gain cell is $i_{out} = i_{in} 10^{-\frac{3}{2} v_{cv}}$. The transfer function of an integrator cell $\alpha$ is thus the following:

\begin{equation}
\alpha(s) = -\frac{1}{RCs} 10^{-\frac{3}{2} v_{cv}}
\end{equation}

We also have: $v_{bp}(s) = v_{hp}(s) \alpha(s)$, and $v_{lp}(s) = v_{hp}(s) \alpha^2(s)$.

The gain of the feedback circuit is noted $\beta$:

\begin{equation}
\beta = \frac{1}{R} 10^{-\frac{3}{2} v_{q}} \times -R = -10^{-\frac{3}{2} v_{q}}
\end{equation}

Since the op-amp has a huge input impedance, we can assume that the current flowing into $i^-$ is null:

\begin{equation}
\frac{v_i(s) - v^-}{R_i} + \frac{v_{hp}(s) - v^-}{R_g} + \frac{v_{lp}(s) - v^-}{R_g} + \frac{v_{lp}(s) - v^-}{R_g} + \frac{v_{bp}(s) \beta - v^-}{R_q} = i^- = 0
\end{equation}


The voltages at the inputs of the op-amp being equal, $v^-(s) = v^+(s) = 0$, hence:

\begin{equation}
\frac{v_i(s)}{R_i} + \frac{v_{hp}(s)}{R_g} + \frac{\alpha^2(s) v_{hp}(s)}{R_g} + \frac{v_{hp}(s) \alpha(s) \beta}{R_q} = 0
\end{equation}

The transfer function for the HP mode can be deduced from there:

\begin{eqnarray}
H_{hp}(s) &=& \frac{v_{hp}(s)}{v_i(s)} \\
 &=& \frac{-1 / R_i}{\frac{1}{R_g} + \frac{\alpha^2(s)}{R_g} + \frac{\alpha(s)\beta}{R_q}} \\
 &=& \frac{-R_g / R_i}{1 + \frac{R_g \alpha(s)\beta}{R_q} + \alpha^2(s)} \\
 &=& \frac{-G}{1 + \frac{R_g \alpha(s)\beta}{R_q} + \alpha^2(s)}
\end{eqnarray}

$G = \frac{R_g}{R_i}$ is the absolute value of the pass-band gain. For further simplifications, we assume $R_g = 2 R_q$.

The transfer function for the LP mode is:

\begin{eqnarray}
H_{lp}(s) &=& \frac{v_{lp}(s)}{v_i(s)} \\
 &=& \frac{v_{hp}(s)}{v_i(s)} \alpha^2(s) \\
 &=& \frac{-G}{\frac{1}{\alpha^2(s)} + \frac{2\beta}{\alpha(s)} + 1} \\
 &=& \frac{-G}{\frac{s^2}{{\left(\frac{1}{RC} 10 ^ {-\frac{3}{2} v_{cv}}\right)}^2} + 2 \frac{s}{{\left(\frac{1}{RC} 10 ^ {-\frac{3}{2} v_{cv}}\right)}} 10^{-\frac{3}{2} v_{q}} + 1}
\end{eqnarray}

By identification with the canonical form of a 2-nd order filter transfer function, this gives the following filter characteristics:

\begin{itemize}
\item Pass-band gain, $-G = -\frac{R_g}{R_i}$
\item Cutoff frequency, $f = \frac{1}{2 \pi R C} 10 ^ {-\frac{3}{2} v_{cv}}$
\item Q factor, $Q = \frac{1}{2} 10^{\frac{3}{2} v_{q}} $

\end{itemize}

Now, let us unleash the power of matplotlib and plot the filter response for several values of Q (rows), for the three modes (columns), and for 5 different cutoff settings (plots within a cell):

\begin{center}
\includegraphics[width=\textwidth]{svf_response.pdf}
\end{center}

\section{Practice}

\subsection{CV tuning}

Because it operates with small supplies ($\pm 5V$) and has its control signals living in the $[0, 5]V$ range, the Shruthi-1 uses the unusual $5V$ = $128$ midi notes scale, rather than the more common $1V$/octave scale -- which would be too narrow in this context. This implies a $2^{\frac{128}{12}} = 1625.5$ ratio between the lowest and highest cutoff frequencies reached by the filter. Using the relationship between cutoff frequency and $v_{cv}$, this implies that $v_{cv}$ swings by 2.141V between its smallest and highest value. The CV output by the Shruthi-1 has a 5V range, so the CV scaling circuit needs to have a $2.141 / 5 = 0.4282$ gain. This is more or less achieved by a $15k\Omega$ / $35k\Omega$ ratio, the $35k\Omega$ being obtained by a $33k\Omega$ resistor in series with a roughly centered $5k\Omega$ trimmer.

\subsection{1-pole filter on resonance CV}

The control signals generated by the Shruthi-1 are PWM modulated ; their 39kHz carrier needs to be removed by a low-pass filter. Cutoff signal conditioning is a serious matter, since the users want to calibrate the cutoff response to get tuned self-oscillation across several octaves. We cannot avoid dedicating PCB space to trimmers and a proper op-amp based cutoff CV scaling circuit.

When it comes to resonance, this is a different matter. The board space was limited and we did not have enough room to implement a proper active filter for smoothing the resonance CV. We went cheap with a passive $RC$ filter ($22k\Omega$, $68nF$, yielding a cutoff frequency of $106 Hz$), but there's a big caveat here! From the SSM2164 datasheet, the CV inputs of the SSM2164 are connected to a $4.5k\Omega$ / $500\Omega$ resistor divider. With the $RC$ filter in place adding some impedance at the input, the $v_{q}$ that the SSM2164 will ``see" is only $\rho = \frac{0.5 + 4.5}{0.5 + 4.5 + 22}$ of the CV output by the Shruthi-1 digital board. The new expression for $Q$ is thus $\frac{1}{2} 10^{\frac{3}{2} \rho v_{q}}$.

So much to save a pair of op-amps! We would have loved using a smaller $R$ to avoid adding to the input impedance of the SSM2164 control inputs, but then the capacitor would have been too large to fit on the board.

\subsection{Pushing self-oscillation}

Self-oscillation occurs when there is no feedback from the BP node to the input of the circuit. Unfortunately, this cannot occur with the circuit of section 1 since it is not possible to entirely mute the SSM2164 gain cell controlling resonance. With the Shruthi-1 Q CV set to a maximum value of $5V$, we get a Q of only $\frac{1}{2} 10^{\rho \frac{3}{2} \times 5} = 12.2$. Still not there! Self oscillation would happen if the signal fed back from the $BP$ node to the input of the filter was null ; but a small portion of it ``bleeds" through the feedback gain cell which is not entirely closed. We can cheat and compensate this bleed by always feeding back a small fraction of the $BP$ node to the input, through a $R_O$ resistor. In this case, the expression for $Q$ becomes:

$$Q = \frac{1}{2} \left(10^{-\frac{3}{2} \rho v_{q}} - \frac{R_Q}{R_O}\right)^{-1}$$

It is now possible to reach self-oscillation even if the feedback gain cell is bleeding a bit. The value of $R_O$ is chosen so that $Q$ can reach 0 before $v_{q}$ reaches $5V$. We observed a compensation of the bleed with values of $R_O$ lower than $680k\Omega$, and picked $R_O = 220k\Omega$ to be on the safe side, even if this means a less gentle ``landing" towards self oscillation.

\subsection{Soft clipping}

Once self-oscillation is reached, the bad surprise is that it crashes on the op-amp rails, yielding a very squelchy sound. Adding a pair of head-to-head Zeners across the BP integrator capacitor ensures that the self-oscillation signal will be soft-limited. We found that a Zener voltage of 4.7V gave the best results with TL07x op-amps (which crash at $\pm 3.6V$ when powered by $\pm 5V$).

\end{document}
